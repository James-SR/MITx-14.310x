\documentclass[]{book}
\usepackage{lmodern}
\usepackage{amssymb,amsmath}
\usepackage{ifxetex,ifluatex}
\usepackage{fixltx2e} % provides \textsubscript
\ifnum 0\ifxetex 1\fi\ifluatex 1\fi=0 % if pdftex
  \usepackage[T1]{fontenc}
  \usepackage[utf8]{inputenc}
\else % if luatex or xelatex
  \ifxetex
    \usepackage{mathspec}
  \else
    \usepackage{fontspec}
  \fi
  \defaultfontfeatures{Ligatures=TeX,Scale=MatchLowercase}
\fi
% use upquote if available, for straight quotes in verbatim environments
\IfFileExists{upquote.sty}{\usepackage{upquote}}{}
% use microtype if available
\IfFileExists{microtype.sty}{%
\usepackage{microtype}
\UseMicrotypeSet[protrusion]{basicmath} % disable protrusion for tt fonts
}{}
\usepackage[margin=1in]{geometry}
\usepackage{hyperref}
\hypersetup{unicode=true,
            pdftitle={Course notes from MITx 14.310x Data Analysis for Social Scientists (EdX)},
            pdfauthor={James Solomon-Rounce},
            pdfborder={0 0 0},
            breaklinks=true}
\urlstyle{same}  % don't use monospace font for urls
\usepackage{natbib}
\bibliographystyle{apalike}
\usepackage{color}
\usepackage{fancyvrb}
\newcommand{\VerbBar}{|}
\newcommand{\VERB}{\Verb[commandchars=\\\{\}]}
\DefineVerbatimEnvironment{Highlighting}{Verbatim}{commandchars=\\\{\}}
% Add ',fontsize=\small' for more characters per line
\usepackage{framed}
\definecolor{shadecolor}{RGB}{248,248,248}
\newenvironment{Shaded}{\begin{snugshade}}{\end{snugshade}}
\newcommand{\KeywordTok}[1]{\textcolor[rgb]{0.13,0.29,0.53}{\textbf{#1}}}
\newcommand{\DataTypeTok}[1]{\textcolor[rgb]{0.13,0.29,0.53}{#1}}
\newcommand{\DecValTok}[1]{\textcolor[rgb]{0.00,0.00,0.81}{#1}}
\newcommand{\BaseNTok}[1]{\textcolor[rgb]{0.00,0.00,0.81}{#1}}
\newcommand{\FloatTok}[1]{\textcolor[rgb]{0.00,0.00,0.81}{#1}}
\newcommand{\ConstantTok}[1]{\textcolor[rgb]{0.00,0.00,0.00}{#1}}
\newcommand{\CharTok}[1]{\textcolor[rgb]{0.31,0.60,0.02}{#1}}
\newcommand{\SpecialCharTok}[1]{\textcolor[rgb]{0.00,0.00,0.00}{#1}}
\newcommand{\StringTok}[1]{\textcolor[rgb]{0.31,0.60,0.02}{#1}}
\newcommand{\VerbatimStringTok}[1]{\textcolor[rgb]{0.31,0.60,0.02}{#1}}
\newcommand{\SpecialStringTok}[1]{\textcolor[rgb]{0.31,0.60,0.02}{#1}}
\newcommand{\ImportTok}[1]{#1}
\newcommand{\CommentTok}[1]{\textcolor[rgb]{0.56,0.35,0.01}{\textit{#1}}}
\newcommand{\DocumentationTok}[1]{\textcolor[rgb]{0.56,0.35,0.01}{\textbf{\textit{#1}}}}
\newcommand{\AnnotationTok}[1]{\textcolor[rgb]{0.56,0.35,0.01}{\textbf{\textit{#1}}}}
\newcommand{\CommentVarTok}[1]{\textcolor[rgb]{0.56,0.35,0.01}{\textbf{\textit{#1}}}}
\newcommand{\OtherTok}[1]{\textcolor[rgb]{0.56,0.35,0.01}{#1}}
\newcommand{\FunctionTok}[1]{\textcolor[rgb]{0.00,0.00,0.00}{#1}}
\newcommand{\VariableTok}[1]{\textcolor[rgb]{0.00,0.00,0.00}{#1}}
\newcommand{\ControlFlowTok}[1]{\textcolor[rgb]{0.13,0.29,0.53}{\textbf{#1}}}
\newcommand{\OperatorTok}[1]{\textcolor[rgb]{0.81,0.36,0.00}{\textbf{#1}}}
\newcommand{\BuiltInTok}[1]{#1}
\newcommand{\ExtensionTok}[1]{#1}
\newcommand{\PreprocessorTok}[1]{\textcolor[rgb]{0.56,0.35,0.01}{\textit{#1}}}
\newcommand{\AttributeTok}[1]{\textcolor[rgb]{0.77,0.63,0.00}{#1}}
\newcommand{\RegionMarkerTok}[1]{#1}
\newcommand{\InformationTok}[1]{\textcolor[rgb]{0.56,0.35,0.01}{\textbf{\textit{#1}}}}
\newcommand{\WarningTok}[1]{\textcolor[rgb]{0.56,0.35,0.01}{\textbf{\textit{#1}}}}
\newcommand{\AlertTok}[1]{\textcolor[rgb]{0.94,0.16,0.16}{#1}}
\newcommand{\ErrorTok}[1]{\textcolor[rgb]{0.64,0.00,0.00}{\textbf{#1}}}
\newcommand{\NormalTok}[1]{#1}
\usepackage{longtable,booktabs}
\usepackage{graphicx,grffile}
\makeatletter
\def\maxwidth{\ifdim\Gin@nat@width>\linewidth\linewidth\else\Gin@nat@width\fi}
\def\maxheight{\ifdim\Gin@nat@height>\textheight\textheight\else\Gin@nat@height\fi}
\makeatother
% Scale images if necessary, so that they will not overflow the page
% margins by default, and it is still possible to overwrite the defaults
% using explicit options in \includegraphics[width, height, ...]{}
\setkeys{Gin}{width=\maxwidth,height=\maxheight,keepaspectratio}
\IfFileExists{parskip.sty}{%
\usepackage{parskip}
}{% else
\setlength{\parindent}{0pt}
\setlength{\parskip}{6pt plus 2pt minus 1pt}
}
\setlength{\emergencystretch}{3em}  % prevent overfull lines
\providecommand{\tightlist}{%
  \setlength{\itemsep}{0pt}\setlength{\parskip}{0pt}}
\setcounter{secnumdepth}{5}
% Redefines (sub)paragraphs to behave more like sections
\ifx\paragraph\undefined\else
\let\oldparagraph\paragraph
\renewcommand{\paragraph}[1]{\oldparagraph{#1}\mbox{}}
\fi
\ifx\subparagraph\undefined\else
\let\oldsubparagraph\subparagraph
\renewcommand{\subparagraph}[1]{\oldsubparagraph{#1}\mbox{}}
\fi

%%% Use protect on footnotes to avoid problems with footnotes in titles
\let\rmarkdownfootnote\footnote%
\def\footnote{\protect\rmarkdownfootnote}

%%% Change title format to be more compact
\usepackage{titling}

% Create subtitle command for use in maketitle
\newcommand{\subtitle}[1]{
  \posttitle{
    \begin{center}\large#1\end{center}
    }
}

\setlength{\droptitle}{-2em}

  \title{Course notes from MITx 14.310x Data Analysis for Social Scientists (EdX)}
    \pretitle{\vspace{\droptitle}\centering\huge}
  \posttitle{\par}
    \author{James Solomon-Rounce}
    \preauthor{\centering\large\emph}
  \postauthor{\par}
      \predate{\centering\large\emph}
  \postdate{\par}
    \date{2018-09-25}

\usepackage{booktabs}

\usepackage{amsthm}
\newtheorem{theorem}{Theorem}[chapter]
\newtheorem{lemma}{Lemma}[chapter]
\theoremstyle{definition}
\newtheorem{definition}{Definition}[chapter]
\newtheorem{corollary}{Corollary}[chapter]
\newtheorem{proposition}{Proposition}[chapter]
\theoremstyle{definition}
\newtheorem{example}{Example}[chapter]
\theoremstyle{definition}
\newtheorem{exercise}{Exercise}[chapter]
\theoremstyle{remark}
\newtheorem*{remark}{Remark}
\newtheorem*{solution}{Solution}
\begin{document}
\maketitle

{
\setcounter{tocdepth}{1}
\tableofcontents
}
\chapter*{Preface}\label{preface}
\addcontentsline{toc}{chapter}{Preface}

The following notes were taken by me for educational, non-commercial,
purposes. If you find the information useful, buy the material/take the
course.

Thank you to the original content providers. Additional ramblings are my
own.

\textbf{\emph{Core Resources}}

\begin{itemize}
\tightlist
\item
  \href{./files/14.310x_3T2018_Schedule.pdf}{Course Schedule}
\item
  \href{./files/14.310x_Grading_and_Homework_Policy__3T2018.pdf}{Grading
  and Homework Policy}
\item
  \href{./files/14310x_Honor_Code_and_Collaboration_Guidelines.pdf}{Honor
  Code and Collaboration Guide}
\item
  \href{./files/Derivation_of_OLS_Estimators.pdf}{Notes - OLS
  Derivation}
\item
  \href{./files/Matrix_Notation_etc.pdf}{Notes - Matrix Notation}
\item
  \href{https://www.rstudio.com/resources/cheatsheets/}{R Studio
  Cheatsheets}
\item
  \href{http://r4ds.had.co.nz/index.html}{R for Data Science Book}
\end{itemize}

\includegraphics[width=1\linewidth]{images/standing}

\chapter{Module 1: Introduction to the
Course}\label{module-1-introduction-to-the-course}

\begin{center}\rule{0.5\linewidth}{\linethickness}\end{center}

\textbf{Module Sections:}

\begin{itemize}
\tightlist
\item
  Welcome to the Course
\item
  Introduction to R
\item
  Introductory Lecture - Data is Beautiful, Insightful, Powerful,
  Deceitful
\item
  Finger Exercises
\item
  Module 1: Homework
\end{itemize}

Module Content:

\begin{itemize}
\tightlist
\item
  \href{./files/M1/Lecture_Slides_01.pdf}{Module 1 Slides}
\item
  \href{./files/M1/M1Paper.pdf}{Homework 1 Background Paper - The
  Persistent Effects Of Peru's Mining Mita}
\item
  \href{./files/M1/R_Instructions.pdf}{R Instructions}\\
\item
  \href{./files/M1/14_310x_Intro_to_R_.zip.pdf}{Intro to R Zip File}\\
\item
  \href{\%22./files/CitesforSara.csv\%22}{Citations Data for Homework 1}
\end{itemize}

\section{Introduction to R}\label{introduction-to-r}

First we setup the environment and install the course files

\begin{Shaded}
\begin{Highlighting}[]
\KeywordTok{library}\NormalTok{(swirl)}
\KeywordTok{install_course_zip}\NormalTok{(}\StringTok{"./files/M1/14_310x_Intro_to_R_.zip"}\NormalTok{,}\DataTypeTok{multi=}\OtherTok{FALSE}\NormalTok{)}
\KeywordTok{swirl}\NormalTok{()}
\end{Highlighting}
\end{Shaded}

IF z is a three number vector e.g.

\begin{Shaded}
\begin{Highlighting}[]
\NormalTok{z <-}\StringTok{ }\KeywordTok{c}\NormalTok{(}\FloatTok{1.1}\NormalTok{, }\DecValTok{9}\NormalTok{, }\FloatTok{3.14}\NormalTok{)}
\end{Highlighting}
\end{Shaded}

If we take the square root of z - 1 and assign it to a new variable
called my\_sqrt e.g.

\begin{Shaded}
\begin{Highlighting}[]
\NormalTok{my_sqrt <-}\StringTok{ }\KeywordTok{sqrt}\NormalTok{(z }\OperatorTok{-}\StringTok{ }\DecValTok{1}\NormalTok{)}
\end{Highlighting}
\end{Shaded}

The result is a vector of length three e.g.

\begin{Shaded}
\begin{Highlighting}[]
\NormalTok{my_sqrt}
\end{Highlighting}
\end{Shaded}

\begin{verbatim}
## [1] 0.3162278 2.8284271 1.4628739
\end{verbatim}

Next, if we create a new variable called my\_div that gets the value of
z divided by my\_sqrt.

\begin{Shaded}
\begin{Highlighting}[]
\NormalTok{my_div <-}\StringTok{ }\NormalTok{z }\OperatorTok{/}\StringTok{ }\NormalTok{my_sqrt}
\end{Highlighting}
\end{Shaded}

The first element of my\_div is equal to the first element of z divided
by the first element of my\_sqrt, and so on\ldots{}

\begin{Shaded}
\begin{Highlighting}[]
\NormalTok{my_div}
\end{Highlighting}
\end{Shaded}

\begin{verbatim}
## [1] 3.478505 3.181981 2.146460
\end{verbatim}

When given two vectors of the same length, R simply performs the
specified arithmetic operation (\texttt{+}, \texttt{-}, \texttt{*},
etc.) element-by-element. If the vectors are of different lengths, R
`recycles' the shorter vector until it is the same length as the longer
vector.

If the length of the shorter vector does not divide evenly into the
length of the longer vector, R will still apply the `recycling' method,
but will throw a warning.

\begin{Shaded}
\begin{Highlighting}[]
\KeywordTok{c}\NormalTok{(}\DecValTok{1}\NormalTok{, }\DecValTok{2}\NormalTok{, }\DecValTok{3}\NormalTok{, }\DecValTok{4}\NormalTok{) }\OperatorTok{+}\StringTok{ }\KeywordTok{c}\NormalTok{(}\DecValTok{0}\NormalTok{, }\DecValTok{10}\NormalTok{, }\DecValTok{100}\NormalTok{)}
\end{Highlighting}
\end{Shaded}

\begin{verbatim}
## Warning in c(1, 2, 3, 4) + c(0, 10, 100): longer object length is not a
## multiple of shorter object length
\end{verbatim}

\begin{verbatim}
## [1]   1  12 103   4
\end{verbatim}

\subsection{Module 1 Homework}\label{module-1-homework}

This is a sample of some of the homework answers. Some questions were
observational or required interpretation of maps for example, as such
these are not inluded here.

\begin{Shaded}
\begin{Highlighting}[]
\KeywordTok{library}\NormalTok{(tidyverse)}
\end{Highlighting}
\end{Shaded}

\begin{verbatim}
## -- Attaching packages ------------------ tidyverse 1.2.1 --
\end{verbatim}

\begin{verbatim}
## v ggplot2 3.0.0     v purrr   0.2.5
## v tibble  1.4.2     v dplyr   0.7.6
## v tidyr   0.8.1     v stringr 1.3.1
## v readr   1.1.1     v forcats 0.3.0
\end{verbatim}

\begin{verbatim}
## -- Conflicts --------------------- tidyverse_conflicts() --
## x dplyr::filter() masks stats::filter()
## x dplyr::lag()    masks stats::lag()
\end{verbatim}

\begin{Shaded}
\begin{Highlighting}[]
\NormalTok{papers <-}\StringTok{ }\KeywordTok{as_tibble}\NormalTok{(}\KeywordTok{read_csv}\NormalTok{(}\StringTok{"./files/M1/CitesforSara.csv"}\NormalTok{))}
\end{Highlighting}
\end{Shaded}

\begin{verbatim}
## Parsed with column specification:
## cols(
##   .default = col_integer(),
##   journal = col_character(),
##   title = col_character(),
##   au1 = col_character(),
##   au2 = col_character(),
##   au3 = col_character(),
##   past5 = col_double(),
##   aflpn90 = col_double(),
##   aulpn90 = col_double(),
##   aulpn80 = col_double(),
##   aulpn70 = col_double(),
##   lcites = col_double()
## )
\end{verbatim}

\begin{verbatim}
## See spec(...) for full column specifications.
\end{verbatim}

Q. 19 Let's take a look at some of the most popular papers. Using the
filter() method, how many records exist where there are greater than or
equal to 100 citations?

\begin{Shaded}
\begin{Highlighting}[]
\CommentTok{#First lets look at our data}
\KeywordTok{head}\NormalTok{(papers)}
\end{Highlighting}
\end{Shaded}

\begin{verbatim}
## # A tibble: 6 x 22
##   journal  year cites title au1   au2   au3   female1 female2 female3  page
##   <chr>   <int> <int> <chr> <chr> <chr> <chr>   <int>   <int>   <int> <int>
## 1 Americ~  1993    31 Jeux~ Kanb~ Keen~ <NA>        0       0      NA    16
## 2 Americ~  1993     4 Chan~ Jame~ <NA>  <NA>        0      NA      NA    22
## 3 Americ~  1993    28 Fact~ Bert~ <NA>  <NA>        0      NA      NA    15
## 4 Americ~  1993    10 Stra~ Garf~ Oh,-~ <NA>        1       0      NA    19
## 5 Americ~  1993     5 Will~ Coat~ Lour~ <NA>        0       0      NA    21
## 6 Americ~  1993    21 Merg~ Kim,~ Sing~ <NA>        0       0      NA    21
## # ... with 11 more variables: order <int>, nauthor <int>, past5 <dbl>,
## #   aflpn90 <dbl>, spage <int>, field <int>, subfld <int>, aulpn90 <dbl>,
## #   aulpn80 <dbl>, aulpn70 <dbl>, lcites <dbl>
\end{verbatim}

\begin{Shaded}
\begin{Highlighting}[]
\KeywordTok{arrange}\NormalTok{(papers,}\KeywordTok{desc}\NormalTok{(cites), title)}
\end{Highlighting}
\end{Shaded}

\begin{verbatim}
## # A tibble: 4,182 x 22
##    journal  year cites title au1   au2   au3   female1 female2 female3
##    <chr>   <int> <int> <chr> <chr> <chr> <chr>   <int>   <int>   <int>
##  1 Econom~  1980  2251 A He~ Whit~ <NA>  <NA>        0      NA      NA
##  2 Econom~  1979  2227 Pros~ Kahn~ Tver~ <NA>        0       0      NA
##  3 Econom~  1987  2164 Co-i~ Engl~ Gran~ <NA>        0       0      NA
##  4 Econom~  1979  1602 Samp~ Heck~ <NA>  <NA>        0      NA      NA
##  5 Econom~  1978  1217 Spec~ Haus~ <NA>  <NA>        0      NA      NA
##  6 Econom~  1982  1077 Auto~ Engl~ <NA>  <NA>        0      NA      NA
##  7 Econom~  1981  1031 Like~ Dick~ Full~ <NA>        0       0      NA
##  8 Econom~  1982   983 Larg~ Hans~ <NA>  <NA>        0      NA      NA
##  9 Econom~  1980   864 Macr~ Sims~ <NA>  <NA>        0      NA      NA
## 10 Econom~  1982   563 Time~ Kydl~ Pres~ <NA>        0       0      NA
## # ... with 4,172 more rows, and 12 more variables: page <int>,
## #   order <int>, nauthor <int>, past5 <dbl>, aflpn90 <dbl>, spage <int>,
## #   field <int>, subfld <int>, aulpn90 <dbl>, aulpn80 <dbl>,
## #   aulpn70 <dbl>, lcites <dbl>
\end{verbatim}

\begin{Shaded}
\begin{Highlighting}[]
\NormalTok{papers }\OperatorTok
\StringTok{  }\KeywordTok{filter}\NormalTok{(cites }\OperatorTok{>=}\StringTok{ }\DecValTok{100}\NormalTok{) }
\end{Highlighting}
\end{Shaded}

\begin{verbatim}
## # A tibble: 205 x 22
##    journal  year cites title au1   au2   au3   female1 female2 female3
##    <chr>   <int> <int> <chr> <chr> <chr> <chr>   <int>   <int>   <int>
##  1 Americ~  1994   117 Is I~ Pers~ Tabe~ <NA>        0       0      NA
##  2 Econom~  1971   149 Furt~ Nerl~ <NA>  <NA>        0      NA      NA
##  3 Econom~  1971   170 The ~ Madd~ <NA>  <NA>       NA      NA      NA
##  4 Econom~  1971   155 Inve~ Luca~ Pres~ <NA>        0       0      NA
##  5 Econom~  1971   139 Some~ Crag~ <NA>  <NA>        0      NA      NA
##  6 Econom~  1971   108 Iden~ Roth~ <NA>  <NA>        0      NA      NA
##  7 Econom~  1972   164 Meth~ Fair~ Jaff~ <NA>        0       0      NA
##  8 Econom~  1972   150 Exis~ Radn~ <NA>  <NA>        0      NA      NA
##  9 Econom~  1973   361 Mani~ Gibb~ <NA>  <NA>        0      NA      NA
## 10 Econom~  1973   107 On a~ Kram~ <NA>  <NA>        0      NA      NA
## # ... with 195 more rows, and 12 more variables: page <int>, order <int>,
## #   nauthor <int>, past5 <dbl>, aflpn90 <dbl>, spage <int>, field <int>,
## #   subfld <int>, aulpn90 <dbl>, aulpn80 <dbl>, aulpn70 <dbl>,
## #   lcites <dbl>
\end{verbatim}

Q.20 Use the group\_by() function to group papers by journal. How many
total citations exist for the journal ``Econometrica''?

\begin{Shaded}
\begin{Highlighting}[]
\NormalTok{papers }\OperatorTok
\StringTok{  }\KeywordTok{group_by}\NormalTok{(journal) }\OperatorTok
\StringTok{  }\KeywordTok{summarise}\NormalTok{(}\KeywordTok{sum}\NormalTok{(cites))}
\end{Highlighting}
\end{Shaded}

\begin{verbatim}
## # A tibble: 7 x 2
##   journal                            `sum(cites)`
##   <chr>                                     <int>
## 1 American-Economic-Review                   3738
## 2 Econometrica                              75789
## 3 Journal-of-Political-Economy               3398
## 4 Quarterly-Journal-of-Economics             8844
## 5 Review-of-Economic-Studies                21937
## 6 Review-of-Economics-and-Statistics         8473
## 7 <NA>                                         14
\end{verbatim}

\begin{Shaded}
\begin{Highlighting}[]
\CommentTok{#or}

\KeywordTok{summarize}\NormalTok{(group_by}
\NormalTok{          (papers, journal), }
          \DataTypeTok{SumOfCites =} \KeywordTok{sum}\NormalTok{(cites))}
\end{Highlighting}
\end{Shaded}

\begin{verbatim}
## # A tibble: 7 x 2
##   journal                            SumOfCites
##   <chr>                                   <int>
## 1 American-Economic-Review                 3738
## 2 Econometrica                            75789
## 3 Journal-of-Political-Economy             3398
## 4 Quarterly-Journal-of-Economics           8844
## 5 Review-of-Economic-Studies              21937
## 6 Review-of-Economics-and-Statistics       8473
## 7 <NA>                                       14
\end{verbatim}

Q.21 How many distinct primary authors (au1) exist in this dataset?

\begin{Shaded}
\begin{Highlighting}[]
\NormalTok{papers }\OperatorTok
\StringTok{  }\KeywordTok{summarise}\NormalTok{(}\KeywordTok{n_distinct}\NormalTok{(au1))}
\end{Highlighting}
\end{Shaded}

\begin{verbatim}
## # A tibble: 1 x 1
##   `n_distinct(au1)`
##               <int>
## 1              2332
\end{verbatim}

\begin{Shaded}
\begin{Highlighting}[]
\CommentTok{#or}

\KeywordTok{n_distinct}\NormalTok{(papers}\OperatorTok{$}\NormalTok{au1)}
\end{Highlighting}
\end{Shaded}

\begin{verbatim}
## [1] 2332
\end{verbatim}

\chapter{Module 2: Fundamentals of Probability, Random Variables, Joint
Distributions + Collecting
Data}\label{module-2-fundamentals-of-probability-random-variables-joint-distributions-collecting-data}

\begin{center}\rule{0.5\linewidth}{\linethickness}\end{center}

\textbf{Module Sections:}

\begin{itemize}
\tightlist
\item
  Fundamentals of Probability
\item
  Random Variables, Distributions, and Joint Distributions
\item
  Gathering and Collecting Data
\item
  Module 2: Homework
\end{itemize}

Module Content:

\begin{itemize}
\tightlist
\item
  \href{./files/M1/Lecture_Slides_02.pdf}{Module 2 Slides - Fundamentals
  of Probability}
\item
  \href{./files/M1/Lecture_Slides_03.pdf}{Module 3 Slides - Random
  Variables, Distributions and Joint Distributions}
\end{itemize}

\section{Fundamentals of Probability}\label{fundamentals-of-probability}

\subsection{Set Theory}\label{set-theory}

\begin{itemize}
\item
  A \emph{sample space} is collection of all possible outcomes
\item
  An \emph{event} is any collection of outcomes - could be one, all or
  none
\item
  If the outcome is a member of an event, the event is said to have
  \emph{occured}
\item
  Event B is said to be \emph{contained} by A, if all outcomes in B also
  are in A
\item
  This is the basis of set theory and used widely in probability,
  although there are some differences between set and probability theory
\item
  If there is no symbol, then this usually means intersection AB in
  probability - in set theory we would write an inverted U e.g.
  \(A \cap B\)
\item
  A and B are mutually exclusive (disjoint in set theory) if they have
  no outcomes in common
\item
  A and B are exhaustive (complimentary in set theory) if their union is
  S (the entire sample space)
\item
  A and B are both mutually exclusive and exhaustive, their union is
  equal to the sample space but they have no events in common - they are
  a partition of the sample space
\end{itemize}

\subsection{Defining Probability}\label{defining-probability}

We assign every event a number P(A) which is the prob. the event will
occur

1 We require that the probability is greater than one for all events in
the sample space - P(A) \textgreater{}= 0 for all A c S 2 The entire
sample space must be equal to one - P(S) = 1 3 For any sequence of
disjoint sets, the prob. of the union of that sequence is equal to the
sum of the probabilities of those events - A\textsubscript{1},
A\textsubscript{2}, \ldots{} , is P(V\textsubscript{i}) =
\(\sum_{i} P(A_i)\)

So we have a sample space, and if it satisfies these three properties,
then we call it a probability. Sometimes this is referred to as a
probability function or a probability distribution, but there is no
standard terminology for all probability theory. Set theory helps to
prove aspects of probability mathematically, for the purposes on this
course, we just need to know what some useful facts are.

\begin{itemize}
\tightlist
\item
  P(A\textsuperscript{c}) = 1 - P(A) =
\end{itemize}

The probability of A compliment, which is the event that contains all of
the outcomes that are not in the event A, the probability of A
compliment is just equal to 1 minus the probability of A. This is useful
if the probability of A comliment (P(A\textsuperscript{c})) is difficult
to compute, where as the probability of A might be very easy to compute.

\begin{itemize}
\tightlist
\item
  \(P (\emptyset)\) =
\end{itemize}

The probability of the empty set is zero.

\begin{itemize}
\tightlist
\item
  If A \textsubscript{c} B then P(A) \textless{}= P(B) =
\end{itemize}

If A is contained in B then the probability of A is less than or equal
to the probability of B

\begin{itemize}
\tightlist
\item
  For all A, 0 \textless{}= P(A) \textless{}= 1 =
\end{itemize}

For any events, the probability of that event is between 0 and 1.

\begin{itemize}
\tightlist
\item
  P(AUB) = P(A) + P(B) - P(AB) =
\end{itemize}

Probability of A union B is just equal to the sum of the probabilities
of those two events minus the probability of the their intersection.

\begin{itemize}
\tightlist
\item
  P(AB\textsuperscript{c}) = P (A) - P(AB) =
\end{itemize}

The probability of A times B complement is equal to the probability of A
minus the probability of the intersection.

\subsection{An example}\label{an-example}

Suppose you have a finite sample space. Let the function n(.) give the
number of elements in a set.

Then define P(A) = n(A)/n(S). This is called a simple sample space, and
it is a probability - we count the number of outcomes and divide by the
number of possible outcomes in the sample space.

We can check that it satisfies the three axioms to ensure it is a
probability:

\begin{enumerate}
\def\labelenumi{\arabic{enumi}.}
\tightlist
\item
  P(A) will always be non-negative because it's a count
\item
  P(S) will equal 1, by definition
\item
  P(AUB) = n(AUB)/n(S) = n(A)/n(S) + n(B)/n(S) = P(A) +P(B).
\end{enumerate}

If you can put your experiment in to this sample space where each
outcome is equally likely, we just need to count to calculate
probabilities of events. So for example, if you want to know how likely
it is you will roll a specific number, say 6, on two dice, we calculate
all the different ways that six occurs then divide this by all possible
outcomes (sample space) - = 5 / 36 = or 13.9\%

\subsection{Another example}\label{another-example}

If the state of Massachusetts issues 6-character license plates, using
one of 26 letters and 10 digits randomly for each character, what is the
probability that I will receive an all digit license plate?

n(S) = 36 (26 + 10) possibilities for each of 6 characters =
36\textsuperscript{6} = 2.176b n(A) = 10 possibilities (for digits only)
for each of 6 characters = 10\textsuperscript{6} = 1m so P(A) = .0005

This is \emph{sampling with replacement}

\emph{What if Massachusetts does not reuse a letter or digit?}

Now, in the sample space, there are 36 possibilities (26 + 10) for the
1st character, 35 left for the 2nd, and so on.

n(S) = 36x35x34x33x32x31 = 36!/30! = 1.402b

Similarly, in the event, there are 10 possibilities for the 1st
character, 9 left for the 2nd, and so on.

n(A) = 10x9x8x7x6x5 = 10!/4! = 151k

so P(A) = 1.402b / 151k = .0001

This is \emph{sampling without replacement}

\subsection{Ordered and Unordered
Arrangements}\label{ordered-and-unordered-arrangements}

In the examples so far, we have used a series of counting rules -
combinatorics i.e.~combinations of objects belonging to a finite set in
accordance with certain constraints.

\begin{enumerate}
\def\labelenumi{\arabic{enumi}.}
\item
  If an experiment has two parts, first one having m possibilities and,
  regardless of the outcome in the first part, the second one having n
  possibilities, then the experiment has m * n possible outcomes - this
  is what we do intuitively
\item
  Any ordered arrangement of objects is called a \emph{permutation}. The
  number of different permutations of N objects is N! The number of
  different permutations of n objects taken from N objects is N!/(N-n)!
  This is the case in the license plate example previously given
\item
  Any unordered arrangement of objects is called a \emph{combination}.
  The number of different combinations of n objects taken from N objects
  is N!/\{(N-n)!n!\}. We typically denote this \(\binom{N}{n}\) - N (big
  objects) choose n (combinations). This is where the order of objects
  doesn't matter i.e.~different orderings don't matter - we take out the
  ordering
\end{enumerate}

So if we had 9 people who each wanted to shake hands, if order doesn't
matter then it is a combination and we take 9 and choose 2 so it
becomes:

9! / \{(9-2)! * 2!\} = 9! / \{7! * 2!\} = 362,880 / \{5,040 * 2\} =
362,880 / 10,080 = 36 combinations

Note, if order did matter and we used the permutations formula the total
would be twice as many

\subsection{Office Arrangements and Pizza
Toppings}\label{office-arrangements-and-pizza-toppings}

Q: If there are six vegetarian pizza toppings and five non-veg, if I
randomly choose two from a hat containing all items, what is the
probability that I end up with a pizza that has one veg and one non-veg
topping?

A:

First we need to count the number of possabilities in the sample space
e.g. \{(V1, V2), (V1, V3), (V1, V4), (V1, N1) \ldots{}\} n(S) =
\(\binom{11}{2}\) = 55 - All outcomes are equally likely

Now we need to define our outcome n(A) = there are A = \{(V1, N1), (V1,
N2), (V2, N1) \ldots{} \} n(A) = 6 * 5 = 30

So the probability is N(A) / n(S) = 30 / 55 = 0.55

In general, I could have chose n toppings and asked what is the
probability that my pizza had n1 vegetarian toppings and n2
non-vegetarian toppings. There would, then, be \(\binom{6}{n_1}\)
possibilities for the veg toppings and \(\binom{5}{n_2}\)for the non-veg
toppings. In other words,

\[P(n_1 veg, n_2  non-veg)=  \binom{6}{n_1} \binom{5}{n_2} \\ \binom{11}{n}\]
This is the basis of the hypergeometric distribution.

\subsection{Independence and Basketball
Example}\label{independence-and-basketball-example}

We call probabilistic events stochastic events. One of the most
fundamental relationships between stochastic events is independence.

\begin{itemize}
\tightlist
\item
  Events A and B are independent if P(AB) = P(A) P(B)
\end{itemize}

That is to say, events A and B are independent if the probability of
their intersection is equal to the product of their probabilities.

\begin{itemize}
\tightlist
\item
  independent events is that knowing one event occurred doesn't give you
  any information about whether the other occurred.
\end{itemize}

This is best represented with an example. If you toss one die, once.
Consider the event, A, that you roll a number less than 5, and the
event, B, that you roll an even number. Are these events independent?

You might consider how could they be, as they rely on the same roll of a
die?

If we use the previous example for independence, we check:

\begin{enumerate}
\def\labelenumi{\arabic{enumi}.}
\item
  Probability of event A is P(A) = 2/3
\item
  Probability of event B is P(B) = 1/2
\item
  Probability of their intersection is P(AB) = 1/3 which is the same as
  P(A) P(B)
\end{enumerate}

So yes, it does satisfy the definition of independence. AB is rolling an
even number less than 5 (e.g.~2 or 4) and P(A)P(B) = P(AB)

\textbf{So knowing one event occurred doesn't give you any information
about whether an other occured}

In another example, if we had a bag of ten poker chips numbered 1 to 10,
with 3 different colours - \(\color{red}{\text{Red(1,2,3,4,5)}}\),
\(\color{blue}{\text{Blue(6,7)}}\) or
\(\color{green}{\text{Green(8,9,10)}}\)

If choosing a poker chip, A that it is blue, and B that it is even,
independent?

\begin{enumerate}
\def\labelenumi{\arabic{enumi}.}
\item
  Probability of event A is P(A) = 2/10 (.2)
\item
  Probability of event B is P(B) = 5/10 (.5)
\item
  Probability of their intersection is P(AB) = 1/10 (or .1) which is the
  same as P(A) P(B)
\end{enumerate}

So yes they are independent, knowing one (that it is blue) does not give
you any information about an other event (it is even).

Note that mutually exclusivity (disjoint events) and independence are
not the same. Mutually exclusive events are not independent, and
independent events cannot be mutually exclusive. Events are mutually
exclusive if P (A and B) = 0.

So our independent events - blue and even - are not mutually exclusive,
they can occur at the same time. Put another way, because events can't
happen at the same time (disjoint or mutually exclusive), they can't be
independent.

So if we take two mutually exclusive events - say the probability of a
poker chip being both green (A) and blue (B) - we can check for the
three parts of independence as:

\begin{enumerate}
\def\labelenumi{\arabic{enumi}.}
\item
  Probability of event A is P(A) = 3/10 (.3)
\item
  Probability of event B is P(B) = 2/10 (.2)
\item
  Probability of their intersection is P(AB) = 0 which is not the same
  as P(A) P(B) (which is 0.06)
\end{enumerate}

As P (AB) = 0 i.e.~they are mutually exclusive they are dependent -
knowing one i.e.~the chip blue DOES give you information about whether
the other event occured - you know it is not green, so the probability
of being green goes from 30\% before being told, to 0\% after being told
it is blue.

\textbf{When events are mutually exclusive, when you know one thing is
true the likelihood of the otehr being true becomes zero}

For more than two events, we define independence the same way - the
events are independent if the probability of their intersection is equal
to the product of their probabilities.

\subsection{Conditional Probability}\label{conditional-probability}

What if knowing one event has occured tells us something about the
probability that another event occured? How can we `update' our
knowledge in the event that the first event has occured?

The probability of A conditional on B is denoted as P(A\textbar{}B). So
the probability of A conditional on B, P(A\textbar{}B), is P(AB)/ P(B),
assuming P(B) \textgreater{} 0. We don't condition on an event if the
probability of an event is 0\%.

So in effect, by knowing one event has occured, it changes or re-defines
our numerator for event B AND it is changing or re-defining our
denominator - the part of the sample space which is now relevant - of
event B.

There is a relationship between indepdence and conditional probability.
Suppose A and B are independent and P(B) \textgreater{} 0. Then,

P(A\textbar{}B) = P(AB)/P(B) = P(A)P(B) (as they are indepdent this is
the same as P(AB)) / P(B) = P(A) (we cancel out P(B) from the previous)

or simply

P(A\textbar{}B) = P(AB)/P(B) = P(A)P(B)/P(B) = P(A)

\subsection{Conditional Probability in American Presidential
Politics}\label{conditional-probability-in-american-presidential-politics}

If candidates for Republican nomination had the following probabilities
- these might be obtained from looking at betting markets

Trump P(A\textsubscript{1}) = .4\\
Cruz P(A\textsubscript{2}) = .3\\
Rubio P(A\textsubscript{3}) = .2\\
Carson P(A\textsubscript{4}) = .1

How can we compute the probability of a Republican win for the
presidency or P(W) i.e.~the general election?

Conditional on winning the nomination, the candidates have following
probabilities of winning the general election:

Trump P(W\textbar{}A\textsubscript{1}) = .25\\
Cruz P(W\textbar{}A\textsubscript{2}) = .2\\
Rubio P(W\textbar{}A\textsubscript{3}) = .6\\
Carson P(W\textbar{}A\textsubscript{4}) = .4

The probability of a Republic win is equal to the probability of the
intersection between a Republican win and the sample space.

The sample space is the union between the four events A1 through A4. A1
through A4 are mutually exclusive and exhaustive events and therefore
form a partition.

In terms of notation, we therefore have:

P(W) = P(WS)

= P(W(A\textsubscript{1} U A\textsubscript{2} U A\textsubscript{3} U
A\textsubscript{4})) because A1-A4 are mutually exclusive and exhaustive
sets, a partition\\
= P(WA\textsubscript{1} U WA\textsubscript{2} U WA\textsubscript{3} U
WA\textsubscript{4})\\
= P(WA\textsubscript{1}) + P(WA\textsubscript{2}) +
P(WA\textsubscript{3}) + P(WA\textsubscript{4})\\
= P(W\textbar{}A\textsubscript{1})P(A\textsubscript{1}) +
P(W\textbar{}A\textsubscript{2})P(A\textsubscript{2}) +
P(W\textbar{}A\textsubscript{3})P(A\textsubscript{3}) +
P(W\textbar{}A\textsubscript{4})P(A\textsubscript{4})

So P(W) = .4x.25 + .3x.2 + .2x.6 + .1x.4 = .32

\subsection{Bayes' Theorem}\label{bayes-theorem}

So far, we have seen that the probability of the intersection between A
and B is equal to the Probability of B conditional on A times the
probability of A:

\begin{itemize}
\tightlist
\item
  P(AB) = P(B\textbar{}A)P(A) = P(A\textbar{}B)P(B)
\item
  provided P(A) \textgreater{} 0 and P(B) \textgreater{} 0 i.e.~both A
  and B have positive probabilities
\item
  so we can write P(A\textbar{}B) = P(B\textbar{}A)P(A)/P(B)
\end{itemize}

We also saw a slightly more complicated version of this, where the
probability of B is the probability of B conditional on A times the
probability of A, plus the probability of B conditional on A complement
times the probability of A complement (note we saw this, albeit with
more compliments, when looking at the Conditional Probability in
American Presidential Politics section)

\begin{itemize}
\tightlist
\item
  P(B) = P(B\textbar{}A)P(A) + P(B\textbar{}Ac)P(Ac)
\item
  P(A\textbar{}B) = P(B\textbar{}A)P(A)/\{P(B\textbar{}A)P(A) +
  P(B\textbar{}Ac)P(Ac)\}
\end{itemize}

C is compliment, and we can do this since A and Ac are partitions of the
sample space S.

A pregnant woman lives in an area where the Zika virus is fairly rare -
1 in 1000 people have it. Still, she's concerned, so she gets tested.
There is a good but not perfect test for the virus---it gives a positive
reading with probability .99 if the person has the virus and a positive
reading with probability .05 if the person does not. Her reading is
positive. How concerned should we be?

P(Z) = .001 (unconditional probability of having Zika) P(Zc) = .999 (999
people don't have it) P(+\textbar{}Z) = .99 (probability of having a
positive test result, conditional on having the zika virus - there is a
1\% change of a false negative) P(+\textbar{}Zc) = .05 (probability of
having a positive result if you don't have the virus is 5\% - false
positive rate) P(Z\textbar{}+) =
P(+\textbar{}Z)P(Z)/\{P(+\textbar{}Z)P(Z) + P(+\textbar{}Zc)P(Zc)\} -
Bayes theorem = .019 - less than 2\% probability

So the introduction of our new data results in us updating our
probability based on the imperfect test, but it doesn't get updated by
much as it still possible it's wrong and the prevelance rate of the zika
virus is rare.

\emph{Example 2}

Assume that the probability of having a rare condition is 1\%. It is
possible to test for the condition, but the test is imperfect. If you
have the condition, there is an 85\% chance that you will test positive.
If you do not have the condition, there is a 5\% chance that you will
test positive. Call the condition C, so that P(C) = 0.01, and call a
positive test t+, so that p(t+\textbar{}C) = 0.85.

What is the probability p(t+) that you test positive for the condition?

So the Probability of having the condition is P(C) 0.01 *
P(t+\textbar{}C) = .85 which is the probability at a test you will test
positive = 0.0085 + P(Cc) * P(t+\textbar{}Cc) = 0.99 * 0.05 = 0.0495 =
0.058

Suppose that you tested positive for the condition. What is the
probability that you truly have the underlying condition?

P(C) = .01 (unconditional probability of having condition) P(Cc) = .99
(99 people don't have it) P(t+\textbar{}C) = .85 (probability of having
a positive test result, conditional on having the condition)
P(t+\textbar{}Cc) = .05 (probability of having a positive result if you
don't have the virus is 5\% - false positive rate) P(C\textbar{}+) =
P(t+\textbar{}C)P(C)/\{P(t+\textbar{}C)P(C) + P(t+\textbar{}Cc)P(Cc)\} -
Bayes theorem = 0.0085 / \{0.0085 + 0.0495\} = .15 - around than 15\%
probability

Suppose that a new test is developed that is more accurate. Now, the
probability of testing positive if you have the condition is 94\%, and
the chance of testing positive if you do not have the condition is only
4\%. Now, what is the probability p(t+) that you test positive for the
condition?

So the Probability of having the condition is P(C) 0.01 *
P(t+\textbar{}C) = .94 which is the probability at a test you will test
positive = 0.0094 + P(Cc) * P(t+\textbar{}Cc) = 0.99 * 0.04 = 0.0396 =
0.049

Suppose that you tested positive for the condition. What is the
probability that you truly have the underlying condition?

P(C) = .01 (unconditional probability of having condition)\\
P(Cc) = .99 (99 people don't have it)\\
P(t+\textbar{}C) = .94 (probability of having a positive test result,
conditional on having the condition)\\
P(t+\textbar{}Cc) = .04 (probability of having a positive result if you
don't have the virus is 5\% - false positive rate)\\
P(C\textbar{}+) = P(t+\textbar{}C)P(C)/\{P(t+\textbar{}C)P(C) +
P(t+\textbar{}Cc)P(Cc)\} - Bayes theorem\\
= 0.0094 / \{0.0094 + 0.0396\}\\
= .19 - around than 15\% probability

Suppose that there is an 80\% chance you will be invited to a dinner
party on a Friday or Saturday evening. In contrast, there is only a 50\%
chance that you will be invited to a dinner party on one of the other
nights of the week. Suppose that you know that you've been invited to a
dinner party tonight, but have forgotten which day of the week it is.
Once you know that you've been invited to a dinner party, what is the
chance that it is either Friday or Saturday? (Please round your answer
to 2 decimal places. For example, if the correct answer is 0.6724,
please input 0.67.)

Hint: Using the notation of Zika question, Let Z:= \{ Fri, Sat\} and
Z\^{}c = \{ M,T,W,Th,Sun\}. Let ``+'' denote invitation. You are given
Pr(``+''\textbar{} Z) = 0.8 and Pr(``+''\textbar{} Z\^{}c) = 0.5. We
want to compute Pr( Z \textbar{} ``+'')

P(Z) = .286 (unconditional probability of it being Friday or Saturday)\\
P(Zc) = .714 (the other 5 days of the week)\\
P(+\textbar{}Z) = .8\\
P(+\textbar{}Zc) = .5\\
P(Z\textbar{}+) = P(+\textbar{}Z)P(Z)/\{P(+\textbar{}Z)P(Z) +
P(+\textbar{}Zc)P(Zc)\} - Bayes theorem\\
= .389 - around 40\% probability

\section{Random Variables, Distributions and Joint
Distributions}\label{random-variables-distributions-and-joint-distributions}

A \emph{random variable} is a real-valued function whose domain is the
sample space - it goes from the sample space to the real line.

A probability goes from the set of all subsets of the sample space in to
the unit interval e.g. {[}0,1{]} between zero and 1

A random variable goes from the sample space to the real line and it has
some numerial charecteristics of the sample space we are interested in.

The probability that something exists induces a distribution of the
random variable, they are not the same.

There are two types of random variable:

\begin{itemize}
\tightlist
\item
  Discrete - one that can take on only a - finite or infinite -
  countably number of values
\item
  Continous - a random variable that can take on any value in some
  interval, bounded or unbounded, of the real line
\end{itemize}

Discrete random variables can be approximated using a continous random
variable, so we typically just use continous. Most of the example we
have seen so far in this section have dealt with discreet random
variables.

\subsection{Probability Functions of Random
Variables}\label{probability-functions-of-random-variables}

For discrete random variables, we often start with a verbal description,
calculate probabilities for each value of the random variable, and then
write down a function or draw a graph describing those probabilities for
different values of the random variable. This is called a probability
function (PF). We saw one of these before in the hypogeometric and
binomial, when looking at the pizza toppings.

Note that:

\begin{itemize}
\tightlist
\item
  The term probability density function is used to draw attention to the
  fact that we are discussing a continuous random variable.\\
\item
  The term probability mass function is used to draw attention to the
  fact that we are discussing a discrete random variable.\\
\item
  The term probability function - or sometimes just the term
  ``distribution'' - is used when we are speaking in more general terms,
  when we're discussing both ``flavors'' of probability function or the
  distinction between the two types of probability functions/random
  variables doesn't matter.
\end{itemize}

Hypergeometric (pizza topping) random variable:

1 Verbal description - Let X be the number of vegetarian toppings I get
on my pizza if I draw the Area Four toppings randomly (without
replacement)

2 Calculation - We can calculate the probability that X = 0, 1, 2, and
so forth, up to the maximum of 6 or n, whichever is smaller, using the
formula from last time. Six is the maximum number of veg toppings
available, n is the number of toppings chosen at random. If there are 0
toppings of a particular type, the result will be undefined, so we
adjust 0! to be defined as just 1. Also, to be consistent with notation
for the random variable, n1 from before now becomes x and since we only
have two options, n2 now becomes n - x

\[P(x  veg, n - x  non-veg)=  \binom{6}{x} \binom{5}{n - x} \\ \binom{11}{n}\]

3 If we then take an example, such as 3 veg toppings - n = 3 - we can
calculate the probabilities for each n

P(X=0) = 6/99\\
P(X=1) = 36/99\\
P(X=2) = 45/99\\
P(X=3) = 12/99

And we can represent the probability function graphically, with points
(aka point mass) then add vertical lines under each point to the axis to
make it easier to read e.g.

\begin{Shaded}
\begin{Highlighting}[]
\KeywordTok{library}\NormalTok{(ggplot2)}

\NormalTok{veggie_choices =}\StringTok{ }\DecValTok{6}
\NormalTok{meat_choices =}\StringTok{ }\DecValTok{5}
\NormalTok{num_toppings =}\StringTok{ }\DecValTok{3}
\NormalTok{veggie_received =}\StringTok{ }\DecValTok{0}\OperatorTok{:}\NormalTok{num_toppings}
\NormalTok{v =}\StringTok{ }\KeywordTok{dhyper}\NormalTok{(}\DataTypeTok{x =}\NormalTok{ veggie_received, }
           \DataTypeTok{m =}\NormalTok{ veggie_choices, }
           \DataTypeTok{n =}\NormalTok{ meat_choices, }
           \DataTypeTok{k =}\NormalTok{ num_toppings)}

\ControlFlowTok{for}\NormalTok{ (i }\ControlFlowTok{in} \DecValTok{1}\OperatorTok{:}\KeywordTok{length}\NormalTok{(v)) \{}
  \KeywordTok{print}\NormalTok{(}\KeywordTok{paste0}\NormalTok{(}\StringTok{"Probability of "}\NormalTok{, }
\NormalTok{               i}\OperatorTok{-}\DecValTok{1}\NormalTok{,}
               \StringTok{" veggie toppings is: "}\NormalTok{,}
               \KeywordTok{round}\NormalTok{(v[i], }\DecValTok{3}\NormalTok{)))}
\NormalTok{\}}
\end{Highlighting}
\end{Shaded}

\begin{verbatim}
## [1] "Probability of 0 veggie toppings is: 0.061"
## [1] "Probability of 1 veggie toppings is: 0.364"
## [1] "Probability of 2 veggie toppings is: 0.455"
## [1] "Probability of 3 veggie toppings is: 0.121"
\end{verbatim}

\begin{Shaded}
\begin{Highlighting}[]
\KeywordTok{ggplot}\NormalTok{(}\DataTypeTok{mapping =} \KeywordTok{aes}\NormalTok{(}\DataTypeTok{x =} \DecValTok{0}\OperatorTok{:}\DecValTok{3}\NormalTok{, }\DataTypeTok{y =}\NormalTok{ v)) }\OperatorTok{+}\StringTok{ }
\StringTok{      }\KeywordTok{geom_point}\NormalTok{(}\DataTypeTok{color =} \StringTok{'red'}\NormalTok{) }\OperatorTok{+}\StringTok{ }
\StringTok{      }\KeywordTok{labs}\NormalTok{(}\DataTypeTok{x =} \StringTok{'Num Veggies'}\NormalTok{, }\DataTypeTok{y =} \StringTok{'Probability'}\NormalTok{) }\OperatorTok{+}\StringTok{ }
\StringTok{      }\KeywordTok{geom_segment}\NormalTok{(}\DataTypeTok{xend =} \DecValTok{0}\OperatorTok{:}\DecValTok{3}\NormalTok{, }\DataTypeTok{yend=}\DecValTok{0}\NormalTok{)}
\end{Highlighting}
\end{Shaded}

\includegraphics{Module2_files/figure-latex/unnamed-chunk-1-1.pdf}

\subsection{The Hypergeometric
Distribution}\label{the-hypergeometric-distribution}

We can represent this in a more general way using notation. We say that
X has a ``hypergeometric distribution with parameters N, K, \& n,''
denoted X \textasciitilde{} H(N,K,n). Where

\begin{itemize}
\tightlist
\item
  N = Total number of toppings
\item
  K = Total number of veg toppings
\item
  n = The number we choose
\end{itemize}

Its Probability Function (PF) is defined similar to before, however we
add a note for which values of x that there is positive probability. We
should, if being fully formal, also add a final part which states it is
0 otherwise. If this is not explicit, as shown below, in terms of the
zero otherwise, we can assume this to be the case.

\[fx(x)=  \binom{K}{x} \binom{N-K}{n - x} \\\binom{N}{n}\]
\[ where \; x = max(0, n + K-N),...,min(n,K) \]

The hypergeometric distribution describes the number of number of
``realized successes'' (in a given sample - represented as x) in n
trials where you're sampling without replacement from a sample of size
N, whose initial probability of success was K/N.

The function provides the probability of X (number of successful
outcomes / number of possible outcomes in the sample space).

\subsection{Steph Curry Shooting
example}\label{steph-curry-shooting-example}

If Steph has a probability of making 44\% of any shot taken and
therefore 56\% chance of missing, we can use the binomial formula to
calculate the probability of making n shots out of 6 possible shots as
follows.

*\href{https://en.wikipedia.org/wiki/Binomial_coefficient}{For more
information see the Binomial Coefficient}

X has a ``binomial distribution with parameters n \& p,'' denoted
\(X \sim B(n,p)\). Its PF is

\[fx(x)=  \binom{n}{x} p^x (1-p)^{n-x} \; \; \; where \; x= 0,1,...n\]

The binomial distribution describes the number of ``successes'' in n
trials where the trials are independent and the probability of success
in each is p.

So plugging in our example we get

\[fx(x)=  \binom{6}{x} .44^x (.56)^{6-x}\]

Which yields:

P(X=0) = .03\\
P(X=1) = .15\\
P(X=2) = .29\\
P(X=3) = .30\\
P(X=4) = .18\\
P(X=5) = .06\\
P(X=6) = .01

As the number of n increases, if p = 50\% (a symetric distribution), the
distribution would begin to look like a normal distribution.

\includegraphics[width=1\linewidth]{images/binomial}

In another example, suppose that you will take 3 penalty kicks in a row.
The likelihood of making each penalty kick is ¾ or 75\%. What is the
probability that you will score 2 (and only 2) of the 3 penalty kicks?

\[fx(x)=  \binom{3}{x} .75^x (.25)^{3-x}\] P(X=0) = .02\\
P(X=1) = .14\\
P(X=2) = .42 \textless{}- this is the answer\\
P(X=3) = .42

\subsection{Properties of the Probability
Distribution}\label{properties-of-the-probability-distribution}

So for a general probability function, we have some broad properties:

\begin{itemize}
\tightlist
\item
  \(0 <= f_x(x_i) <= 1\) which is to say the value of any probability
  function is going to be between 0 and 1
\item
  \(Σ_i f_x_ (x_i) = 1\) if you sum up over all of the possible values
  it will sum to 1
\item
  \(P(A) = P(XcA) = Σ_Af_x(x_i)\) which is to say if you want the
  probability over a set of values of x, you just sum up the individual
  values for each item in the set
\end{itemize}

For a continous random variable, we rarely start with a verbal
description. Instead, we typically have a density that describes the
probability that the random variable is in various regions. The density,
or probability density function (PDF) is the continuous compliment to
the discrete PF. The PF (discret) and PDF (continous) are similar but
not exactly the same.

A random variable X is continuous if there exists a nonnegative function
f\_X\_ such that for any interval A c R as follows. We tend to speak
about a region, that A is in a region of the real line (R), the
probability that X is in A is equal to the integral over that region A
of the PDF.

\[P(X c A) = \int_{A} f_X(x)dx\]

\bibliography{book.bib}


\end{document}
